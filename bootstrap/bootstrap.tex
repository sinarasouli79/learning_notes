\documentclass{article}
\usepackage[localise=on]{xepersian}
\settextfont{XB Yas}
\begin{document}
	\عنوان{بوت استرپ مهران تعریف}
	\عنوان‌ساز
	\صفحه‌جدید
	\فهرست‌مطالب
	\صفحه‌جدید
	\قسمت{قسمت اول شروع به کار با بوت استرپ}
		برای نصب یا فعال سازی بوت استرپ لینک سی‌دی‌ان سی‌اس‌اس و جاوا اسکریپتشو از خود وبسایت به فایل اچ تی ام ال اضافه میکنیم
		
	\قسمت{قسمت دوم آشنایی با بخش های مختلف بوت استرپ}
		\زیرقسمت{بوت استرپ راست چین}
		از داکیومنت لینک هارو کپی کن
		
	\قسمت{قسمت سوم کار بایوتیلیتی ها}
		\زیرقسمت{‌Background}
		بوت استرپ یک سری رنگ هارو داره که با اونا کار میکنه اگر بخوان بکگراند عوض بشه رنگش از اونا استفاده میکنه\\
		لایت روشن هست بادی رنگ بادی رو میگیره وایت هم سفیده ترنسپرنت هم که شفاف هست\\
		\زیرقسمت{رنگ متن}
		مثل همون بکگراند هست ولی اینطوری تسکست-رنگ\\
		\زیرقسمت{تمرین}
			۱۱ رنگ پس زمینه را در کلاس کانتینر بامارجین مایین ۵ نمایش بدید
		\زیرقسمت{Gradient} به معنی شیب یا تدارج رنگی هست برای گرادیانت پس زمینه از این کلاس استفاده میکنیم \متن‌لاتین{bg-gradient}\\
		حالت گرادیانت و عادی یک رنگ رو باهم مقایسه کنید
		
		\زیرقسمت{Opacity} شفافیت هست هرچه عدد به یک نزدیک تر باشه یا درصد کمتر باشه شفافیت بیشتر هست  از کلاس \متن‌لاتین{bg-opacity-number} 
		بجای نامبر از عدد های ۷۵، ۵۰، ۲۵ و ۱۰ 
		\پاراگراف{تمرین} طیف ها متفاوت یک رنگ را با اپسیتی بسازید از ۱۰۰ تا ۱۰ \\
		
		\زیرقسمت{text-colors} همون رنگ ها برای متن هم داریم \متن‌لاتین{text-color} یک رنگ متفاوت که اینجا هست \متن‌لاتین{text-muted} هست که 
		یک رنگ تیره و مات هست استفادش هم زیاده معمولا یک تایتل اچ ان بایک میتوتد پاراگراف میاد که لید هست تمرینش کن
		
		\زیرقسمت{text-opacity} مثل همون اوپسیتی برای بکگراند هست فقط ۱۰ نداره دیگه تمرین کن روی یک یک طیف رنگ(4)
		
		\زیرقسمت{Borders} اگر از کلاس بردر استفاده کنیم اون باکسی که داریم دورش یک بردری میندازه میتونیم با ستارت و تاپ و اند و باتن
		
		\زیرزیرقسمت{حذف یک بردر}
		بردر های مختلف بدیم اگر بخواهیم فقط یک طرف بردر نداشته باشه از این کلاس ها استفاده میکنیم
		\متن‌لاتین{border-\{start top end botton\}-0} فقط باید کلاس بردر هم بهش بدیم
		
		\زیرزیرقسمت{border-color} برای رنگ دادن هم به بردر از کلاس بردر-کالر استفاده میکنیم
		
		\زیرزیرقسمت{border-width} برای ضخامت از این کلاس استفاده می کنیم از یک تا ۵ \متن‌لاتین{border-\{1 2 3 4 5\}}
		
		\زیرزیرقسمت{border radius} از کلاس rounded استفاده میکنیم میتونیم برای هر طرف استفاده بشه \\
		\متن‌لاتین{rounded\{start top botton end\}} برای دایره از \متن‌لاتین{rounded-circle} و برای بیضی از \متن‌لاتین{border-pill}\\
		میتوانیم با اعداد ۰ تا ۳ میزان گرد بودن هم مشخص کنیم \متن‌لاتین{rounded-\{0 1 2 3\}}
		
		
		\زیرقسمت{display}
		برای دیسپلی از یکی از این کلاس ها بعد از \متن‌لاتین{d-} استفاده میکنیم:\\ 
		\متن‌لاتین{none, inline, inline-block, block, grid, table, table-cell, table-row, flex, inline-flex}\\
		
\end{document}