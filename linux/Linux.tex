\documentclass{book}
\usepackage{longtable}
\usepackage{hyperref}

\begin{document}
	\title{My Linux Notes}
	\maketitle
	\author{Overleaf}
	\pagebreak
	\tableofcontents
	\pagebreak
	\chapter{Managing Software and Processes}
	\section{Arjang}
	static linking no dependencies, too space, no library sharing(redundent) wulnerable, secure bug in library  or feature\\
	dynamic linking not bundelling library inside program\\
	install dependencies first , no redandent files, dependencies in dependencies, dependency hell \\
	package: archive file(like zip, cpio, tar)
	package format: rpm, deb \\
	rpm and deb are the same both of them are package\\
	extract package -> install package\\
	alien convert rmp to deb and other\\
	compile source and use : system doesn't know it (upgrade and updating is manual config file)\\
	packager: compile app to create package and then download the package \\
	developer and packager are not same necessary\\
	distributor repository  , website rpmfind.net, 
	not complied package text or script\\
	extract manually package: there is not only one package, package is expload in the system and file system hierarchy \\
	uninstall package: remove all package file from the system\\
	to solve this problem use package manager apps: it has a database to keep the where and when the package installed\\
	its easy to remove package, we can query, information\\
	rpm linux package manager in redhat base distributions\\
	package information inside it: writer, packager, dependencies, scrip for after and before installing and deleting\\
	rpm -i install\\
	rpm -q query on rpm installed app database \\
	rpm -qa all package in this system\\
	update kernel -> download package\\
	rpm -q package name if package installed we have output\\
	the package name and the command name not necessary the same\\
	ls, cat, ... -> core utils\\
	rpm -ql list of a package's files\\
	rpm -ql coreutils or jdate\\
	rpm -qi package information \\
	rpm -qR Required dependencies of a package\\
	rpm -q --what requires package name reverse dependencies\\
	rpm -e package erase a package\\
	rpm -e --nodeps package name force to delete package if there's a dependency error\\
	rpm -qpr query from package file instead of installed app database\\
	rpm -qf /file/name show package of the file\\
	extract a file to install \\
	rpm2cpio | cpio -id\\
	
	
	
	\section{Looking at Package Concepts}
	\section{Using RPM}
	Redhat Package Manager\\
	\subsection{RPM Distributions and Conventions}
	package-name-version-release-architecture.rpm\\
	different distribution may have different package name for the same program\\
	release also called build number\\	
	\subsection{The rpm Command Set}
	\begin{tabular}{lll}
		\hline
		Short & Long      & Description                                \\ \hline
		-e    & --erase   & removes the specified package              \\
		-i    & --install & install the specified package              \\
		-q    & --query   & queries the specified package is installed \\
		-U    & --upgrade & install of upgrades the specified package  \\
		-Uvh  &           & installing/upgrading and rpm package file h for hash signs\\
		-k & checksig & check the signatures\\
		-V & --verify & check the installation is correct or not man rpm -V \\
		- & & \\
	\end{tabular}
		rpm -uvh *.rpm
	\subsubsection{Querying RMP Packages}
		use the -q action to perform a simple query on the package management database for the installed packages
		\begin{tabular}{lll}
			\hline
			Short & Long            & Description                                                                                           \\ \hline
			-c    & --configfiles   & list the names and the absolute directory references of package configuration files                   \\
			-i    & --info          & provides detailed information, including version, installation date and signatures                    \\
			-l && \\
			-f && \\
			N/A   & --provides      & show the facilities package providers                                                                 \\
			-R    & --requires      & display various package requirements(dependencies)                                                    \\
			-s    & --state         & provides state of different files in a package, such as normal(installed), not installed, or replaced \\
			N/A   & --what-provides & show what package a file belongs
		\end{tabular}
	\subsection{Extracting Data from RPMs}
	\subsection{Using YUM}
	\subsection{Using ZYpp}
	\subsection{Using DNF}
		\href{https://dnf.readthedocs.io/en/latest/use\_cases.html}{dnf document}
	
	\section{VIM}
		\subsection{Commands}
		\begin{longtable}{ll}
			\hline
			Help                                        &                                                   \\ \hline
			help command                                & help for the command                              \\
			:q                                          & close the help window                             \\
			.                                           & repeat the last change                            \\ \hline
			Exit Commands                               &                                                   \\ \hline
			:w                                          & save the file                                     \\
			:q                                          & exit (if didn't change anything)                  \\
			:q!                                         & exit without saving                               \\
			:wq                                         & save and exit(:x ZZ)                              \\
			:wq file                                    & save ass                                          \\ \hline
			Insert Mode                                 &                                                   \\ \hline
			i                                           & insert mode                                       \\
			o                                           & create a line below                               \\
			O                                           & create a line above                               \\
			a                                           & append text after cursor                          \\
			A                                           & append text at the end of the line                \\ \hline
			Movement Commands                           &                                                   \\ \hline
			h                                           & left                                              \\
			l                                           & right                                             \\
			j                                           & down                                              \\
			k                                           & up                                                \\
			0(hat)                                      & begging of the line                               \\
			\$                                          & end of the line                                   \\
			w                                           & move forward word by word(W use space only)       \\
			b                                           & move backward word by word(B use space only)      \\
			e                                           & end of word                                       \\
			:n                                          & jump at line number                               \\
			G :                                         & end of file                                       \\
			ctrl + f                                    & page down                                         \\
			ctrl + b                                    & page up                                           \\ \hline
			Deleting Text                               &                                                   \\ \hline
			x:                                          & delete under cursor                               \\
			X:                                          & delete before cursor                              \\
			dw:                                         & delete word                                       \\
			diw:                                        & delete inner word                                 \\
			D:                                          & clear line                                        \\
			(n)dd:                                      & delete n line                                     \\
			dG:                                         & delete entire line below                          \\ \hline
			Replacing                                   &                                                   \\ \hline
			r char                                      & replace char with under cursor char               \\
			R                                           & Replacing Mode                                    \\ \hline
			Copy \& Paste                               &                                                   \\ \hline
			(n)yy                                       & yank yank(copy) n line                            \\
			p                                           & paste blow                                        \\
			P                                           & paste above                                       \\
			J                                           & join                                              \\ \hline
			Visual Mode                                 &                                                   \\ \hline
			v                                           & visual mode (all movement commands works here)    \\
			y                                           & copy                                              \\
			d                                           & delete and copy(cut)                              \\ \hline
			Searching                                   &                                                   \\ \hline
			/pattern                                    & search for pattern                                \\
			/enter(n)                                   & next result                                       \\
			?enter(N)                                   & previous result                                   \\
			/                                           & search top to bottom                              \\
			?                                           & search bottom to top                              \\ \hline
			Search \& Replace                           &                                                   \\ \hline
			[range]s/{pattern}/{string}/[flags] [count] & replace string with patter\footnote{defaults=1}   \\
			s/{pattern}/{string}/g                      & replace string with all pattern in a line         \\
			\% s/{pattern}/{string}/                    & replace string with the first pattern whole file  \\
			\% s/{pattern}/{string}/g                   & replace string with all pattern in whole file     \\ \hline
			Configurations                              &                                                   \\ \hline
			configuration file                          & \(\sim \)/.vimrc                                  \\
			"                                           & comment in vimrc                                  \\
			syntax[on/off]                              & language syntax                                   \\
			set [nu/nonu]                               & line number                                       \\
			set [wrap/nowrap]                           & breaking the line                                 \\
			set ignorecase                              & (search and replace by default is case sensitive)
		\end{longtable}				
	\section{Links}
		\subsection{Hard Links}
		\subsection{Soft(Symbolic) links}

	\section{TO-DO}
		TERM \\
		
\end{document}
