\documentclass{article}
\usepackage[localise=on]{xepersian}
\settextfont{XB Yas}
\begin{document}
	\قسمت{اندیشه نقادانه (شاهین و وریا) قسمت ۱}
	\پاراگراف{فلسفه} چیزی جز پرداختن به سوالات و پرسش های اصلی هستی و زندگی ما نیست پس فقط تغییر مهم نیست تفسیر مهم است
	\زیرقسمت{اندیشه نقادانه}
		\پاراگراف{اندیشه نقادانه چیست؟} یک ابزار برای اینکه بفهمیم چطور با پرسش ها برخورد بکنیم و چطور پرسش کنیم و برای تشخیص استدلال ها و گزاره ها
		
		\پاراگراف{اگر نظم سیاسی امریکا اجازه اعتراض و نقد مدنی داده چرا اعتراضات هنوز به خشونت کشیده می‌شود؟}
		مصادره به مطلوب؛در جهانی که اجازه اعتراضات مدنی داده میشود اعتراضات به خشونت کشیده نمیشود در امریکا اعتراضات به خشونت کشیده میشود پس اجازه اعتراض مدنی داده نشده.\\
		این صحیح نیست چون اجازه اعتراضات مدنی و به خشونت کشیده شدن اعتراضات ملزوم یکدیگر نیستند میشود اجازه اعتراضات مدنی داده شود ولی باز هم اعتراضات به خشونت کشیده شود مثل لندن سال ۲۰۱۱ که اقای مراد فرادپور میگف شاپینگ سوسیالیستی\\
		
\پاراگراف{سوال سامان داده شده اینگونه میتواند باشد} چرا باوجود اجازه اعتراضات مدنی هنوز اعتراضات در آمریکا به خشونت کشیده میشود؟ اینجا یکم جا میگذاریم یعنی ممکن هست دلایل دیگری هم دخیل باشد در این موضوع

	\پاراگراف{مثال ایران} چرا در ایران ناموس کشی اتفاق می‌افتد؟ چون آزادی نیست، چون قانون نیست، چون نظام ناکارآمده،...۱۹:۰۰
	
	
	\پاراگراف{در رابطه با قضیه آمریکا} میتونه این از نقص سیستم امریکا باشه چون امریکا و غرب از هارمونی فهم داره این اجازه میده که مخالف خودش بیاد درون خودش اما به یک شرط که استحاله بشه باید بقا در انجا را بپذیرد برای همین بهشت ما ایرانی ها عرب ها چینی ها و هر جای دیگه که ازادی نیست میشود این امکان باعث میشه که اگر خشونتی هم بخاطر نقص سیستم اتفاق بافتد اما امار کشته شده ها کمتر شود این یعنی سیستم داره تیزهوش میشه و خودش رو گسترش میده
	\پاراگراف{استدلال:} قانون شکنی مشکلی ندارد برای چه میگویید قانون شکنی بد است؟ اگه اینطوری هست پس سافرجت ها زنانی که برای حقوقشون ریختن
	بیرون هم کارشون اشتباه بوده چون قانون شکنی بوده این مغالطه مثال اشتباه است یعنی من موقعتی بیان میکنم بعد میگم الان هم مثل اون موقعیت
	هست پس مثل همون رفتار کنیم ولی درواقع موقعیت ها یکی نیست اون زنان برای حق نداشته خودشون در قانون اعتراض میکردند ولی سیاه پوستان الان
	درقانون اساسی امریکا حق دارند و یک مغالطه دیگر استاندارد دوگانه الان در لیبی یک بازار برده فروشی هست چرا به اون اعتراض نمیکنید و یک 
	استدلال دیگه چون فردی در گذشته نژاد پرست بوده باید مجسمشو خراب کنیم پس نصف تاریخ باید بریزیم دور از نادر شاه باید مجسمه بکشیم پایین
	باید کاخ گلستان خراب بکنیم، پرسپولیس باید خراب کنیم چون خشایار شاه اون بالاست و ...
	
	\پاراگراف{مغالطه توصیف در جایگاه توجیه} توصیف یک پدیده اون رو توجیه نمیکنه، توجیه یعنی این که ما وجوه متفاوت یک چیز رو ببینیم
	در سطوح متفاوت از زوایای متفاوت
	\پاراگراف{نکته} اندیشه نقادانه یعنی شناخت سره از ناسره دکارت میگه من فکر میکنم پس هستم
	
	\پاراگراف{باور} موتور راه اندازنده افکار ما و افعال ما سوال ایا برای باور هایمان دلایل موجه داریم ؟ اندیشه نقادانه به ما کمک میکند
	بفهمیم (نکته فلاسفه امدند جهان را تفسیر کنند ما امدیم تغییرش بدیم) 
	
	\پاراگراف{چرا آتنا امشب مهمونی نیامده؟} ۱:کسی میگه اتنا را دوست ندارد؛ غیرموجه این دلیل خوبی نیست چون یک نفر از آتنا خوشش نمی‌آید دلیل
	نمیشه که اون مهمونی نیاد ولی اگر اون شخص صاحب مهمانی باشه این دیگه موجه است ۲: میگن اتنا خجالتی از مهمونی زیاد خوشش نمیاد این میتونه
	احتمال رو زیاد کنه ۳: میگن اتنا در پاریس است این تقریبا به طور حتم باید قبول کنیم که نمیشه اتنا یک شبه بیاد (این نوع از گفت و گو 
	رو ما ار از افلاطون داریم درواقع با اون شخصیتی که در جمهوری داره به نام سقراط)
	\پاراگراف{نکته میشود داده های حقیقی داد ولی نتایج جانب دارانه گرفت} مثلا بگیم شخصی کتاب خوان هست چون کتاب خوان هست فهم لازم در بعضی 
	امور را دارد چون فهم لازم را دارد پس میتواند یک جامعه را اداره کند در تمام این مراحل ایراد منطقی وجود دارد شاید ایشون کل عمر کتاب 
	داستان خونده، چون فهم لازم در بعضی از امور را دارد به این معنی نیست که میتواند جامعه را اداره کند ارسطو این بحث هارا درکتاب منطقش
	ریتادیکس اینهارا لیست کرد
	
	\پاراگراف{ذهن و زبان} ما معتقد بودیم که ذهن ما زبان مارو میسازه ولی ویتگنشتاین اومد گفت نه زبان هست که ذهن مارو میسازه(بحث طولانیه) 
	پس اگر خیلی مغالطات رو تکرار کنیم ذهن ما همش میشه مغالطات ولی در مطلب فلسفی اگر یک مغالطه هم بکنیم مطلب بیهوده میشه چون پازل فکری 
	بهم میریزه مگر اینکه اون گوشه های مطلب باشه
	
	\پاراگراف{فلسفه زبان و واژه شناسی} مدیون نیچه هستیم و بعد سوفستایی ها (پیشا سقراطی ها) که به همه چیز در جهان زبان نگاه میکردند پس
	واژه ها معانی متفاوتی میتوانند داشته باشند مثلا اخلاق برای ما در زبان فارسی فرق داره با واژه مثل اخلاق در زبان عربی برای یک عرب زبان
	چون زبان یک تاریخ داره یک سنت داره و طبق این سنت یک مفهوم تاریخی هم براش ساخته شده خیلی وقتا ما اینارو به عنوان غلط مصطلح استفاده 
	میکنیم مثلا عاقبت گرگزاده گرگ شود گرچه با ادم بزرگ شود یا تربیت نا اهل را چون گردکان بر گنبد است یعنی کسی که بده ذاتش بده یک نگاه
	ذاتگرایانه و اشتباه غلط  که کل روانشناسی میبره زیر سوال اقا شما حاصل بچگی و تحریکات و نیازهای خودتی
	\پاراگراف{مغالطه علت اشتباه } یعنی ما علت اشتباهی رو به یک پدیده نسبت بدیم فلانی خوبه چون خانوادش خوبن هنرمنده چون خانوادش هنرمندن
	چه ربطی داره خانواده میتواند یک عامل مهم باشد ولی تعیین کننده نیست ما دنبال علل مختلف باید باشیم
	
	\زیرقسمت{مغالطه مصادره به مطلوب \متن‌لاتین{Begging the Question}} کسی یک ادعا میکند برای اثبات ادعا یک سری دلایل میارند که در اون دلایل
	انگار از قبل ادعا به عنوان اصل پذیرفته شده 
	\پاراگراف{مثال}
	از کجا میدونی خدا وجود داره\\
	در کتاب مقدس گفته\\
	از کجا میدونی کتاب مقدس درست میگه؟\\
	چون خدا فرستاده\\
	همه عروسک ایرون من رو دارن میخرن\\
	چرا؟\\
	چون پرفروشترین عروسک ساله:|\\
	شک فلسفی دکارت: نمیتونیم مطمئن باشیم خوابیم یا بیدار مثل فیلم ماتریکس شما میتونی مغزی باشی در خمره :)\ \\
	ولی یکی میاد میگه من دستمو دارم میبینم دیگه من دارم میبینم این جهان واقعیه درواقع دکارت میخواد ارتباط بین احساس رو از بین ببره پس 
	نمیتوانیم با احساس احساس رو ثابت کنیم
	
	\زیرقسمت{مغالطه حمله به شخص \متن‌لاتین{Ad Hominem}} بجای اینکه به فکر به دلیل اشاره بشه به شخص اشاره میشه
	\پاراگراف{مثال} اقا تو میگی عراقی شاعر بزرگی بوده همه میدونن این ادم مشکل اخلاقی داشته، هنر را باید فارق از شخصیت هنرمند نگاه کرد\\
	اقا چی میگی هی مصدق مصدق جلال‌الدین فارسی گفته این به ما خمس و ذکات میداده اینکه چون تو مشکل شخصی با خمس و ذکات و اسلام داری دلیل 
	نمیشه مصدق ادم بدی باشه \\ اقا تو چی هی میگی اختلاف طبقاتی تو که حرف کمنیست ورشکسته هارو میزنی توهم از اونایی ارهههه ؟؟ :)))\\
	\زیرزیرقسمت{انواع مختلف} مثلا شخص معروفی یک حرفی زده:‌ اقا شما بی خدا ها چی میگید مگه نیوتن نمیشناسی این خدا رو قبول داشته پس چی میگی\\
	یا مثلا هیچ کس از متفکرین نفهمیدن تو فهمیدی؟\\
	
	\پاراگراف{ببین چه میگوید نه اینکه که میگوید} اگه یکی میگه سیگار کشیدن بده ببین چرا میگه بده نگو خودتم که سیگار میکشی ممکنه یک فرد 
	رو به ریاکاری متهم کنیم مثلا هیتلر اگه بگه ازادی
	
	
	
	
	
	
	
	
	\قسمت{مغالطه در استدلال(۳) مغالطات ربطی}
	\زیرقسمت{مغالطه مصادره‌به‌مطلوب}
	نام لاتین \شروع{لاتین} begging the question\پایان{لاتین}
	نتیجه استدلال که باید اثبات شود،‌ در فرض اثبات شده در نظر گرفته شود.\\
	
	
\end{document}