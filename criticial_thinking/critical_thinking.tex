\documentclass{article}
\usepackage[localise=on]{xepersian}
\settextfont{XB Yas}
\begin{document}
	\قسمت{اندیشه نقادانه (شاهین و وریا) قسمت ۱}
	\پاراگراف{فلسفه} چیزی جز پرداختن به سوالات و پرسش های اصلی هستی و زندگی ما نیست پس فقط تغییر مهم نیست تفسیر مهم است
	\زیرقسمت{اندیشه نقادانه}
		\پاراگراف{اندیشه نقادانه چیست؟} یک ابزار برای اینکه بفهمیم چطور با پرسش ها برخورد بکنیم و چطور پرسش کنیم و برای تشخیص استدلال ها و گزاره ها
		
		\پاراگراف{اگر نظم سیاسی امریکا اجازه اعتراض و نقد مدنی داده چرا اعتراضات هنوز به خشونت کشیده می‌شود؟}
		مصادره به مطلوب؛در جهانی که اجازه اعتراضات مدنی داده میشود اعتراضات به خشونت کشیده نمیشود در امریکا اعتراضات به خشونت کشیده میشود پس اجازه اعتراض مدنی داده نشده.\\
		این صحیح نیست چون اجازه اعتراضات مدنی و به خشونت کشیده شدن اعتراضات ملزوم یکدیگر نیستند میشود اجازه اعتراضات مدنی داده شود ولی باز هم اعتراضات به خشونت کشیده شود مثل لندن سال ۲۰۱۱ که اقای مراد فرادپور میگف شاپینگ سوسیالیستی\\
		
\پاراگراف{سوال سامان داده شده اینگونه میتواند باشد} چرا باوجود اجازه اعتراضات مدنی هنوز اعتراضات در آمریکا به خشونت کشیده میشود؟ اینجا یکم جا میگذاریم یعنی ممکن هست دلایل دیگری هم دخیل باشد در این موضوع

	\پاراگراف{مثال ایران} چرا در ایران ناموس کشی اتفاق می‌افتد؟ چون آزادی نیست، چون قانون نیست، چون نظام ناکارآمده،...۱۹:۰۰
	
	
	\پاراگراف{در رابطه با قضیه آمریکا} میتونه این از نقص سیستم امریکا باشه چون امریکا و غرب از هارمونی فهم داره این اجازه میده که مخالف خودش بیاد درون خودش اما به یک شرط که استحاله بشه باید بقا در انجا را بپذیرد برای همین بهشت ما ایرانی ها عرب ها چینی ها و هر جای دیگه که ازادی نیست میشود این امکان باعث میشه که اگر خشونتی هم بخاطر نقص سیستم اتفاق بافتد اما امار کشته شده ها کمتر شود این یعنی سیستم داره تیزهوش میشه و خودش رو گسترش میده
	\پاراگراف{استدلال:} قانون شکنی مشکلی ندارد برای چه میگویید قانون شکنی بد است؟ اگه اینطوری هست پس سافرجت ها زنانی که برای حقوقشون ریختن
	بیرون هم کارشون اشتباه بوده چون قانون شکنی بوده این مغالطه مثال اشتباه است یعنی من موقعتی بیان میکنم بعد میگم الان هم مثل اون موقعیت
	هست پس مثل همون رفتار کنیم ولی درواقع موقعیت ها یکی نیست اون زنان برای حق نداشته خودشون در قانون اعتراض میکردند ولی سیاه پوستان الان
	درقانون اساسی امریکا حق دارند و یک مغالطه دیگر استاندارد دوگانه الان در لیبی یک بازار برده فروشی هست چرا به اون اعتراض نمیکنید و یک 
	استدلال دیگه چون فردی در گذشته نژاد پرست بوده باید مجسمشو خراب کنیم پس نصف تاریخ باید بریزیم دور از نادر شاه باید مجسمه بکشیم پایین
	باید کاخ گلستان خراب بکنیم، پرسپولیس باید خراب کنیم چون خشایار شاه اون بالاست و ...
	
	\پاراگراف{مغالطه توصیف در جایگاه توجیه} توصیف یک پدیده اون رو توجیه نمیکنه، توجیه یعنی این که ما وجوه متفاوت یک چیز رو ببینیم
	در سطوح متفاوت از زوایای متفاوت
	\پاراگراف{نکته} اندیشه نقادانه یعنی شناخت سره از ناسره دکارت میگه من فکر میکنم پس هستم
	
	\پاراگراف{باور} موتور راه اندازنده افکار ما و افعال ما سوال ایا برای باور هایمان دلایل موجه داریم ؟ اندیشه نقادانه به ما کمک میکند
	بفهمیم (نکته فلاسفه امدند جهان را تفسیر کنند ما امدیم تغییرش بدیم) 
	
	\پاراگراف{چرا آتنا امشب مهمونی نیامده؟} ۱:کسی میگه اتنا را دوست ندارد؛ غیرموجه این دلیل خوبی نیست چون یک نفر از آتنا خوشش نمی‌آید دلیل
	نمیشه که اون مهمونی نیاد ولی اگر اون شخص صاحب مهمانی باشه این دیگه موجه است ۲: میگن اتنا خجالتی از مهمونی زیاد خوشش نمیاد این میتونه
	احتمال رو زیاد کنه ۳: میگن اتنا در پاریس است این تقریبا به طور حتم باید قبول کنیم که نمیشه اتنا یک شبه بیاد (این نوع از گفت و گو 
	رو ما ار از افلاطون داریم درواقع با اون شخصیتی که در جمهوری داره به نام سقراط)
	\پاراگراف{نکته میشود داده های حقیقی داد ولی نتایج جانب دارانه گرفت} مثلا بگیم شخصی کتاب خوان هست چون کتاب خوان هست فهم لازم در بعضی 
	امور را دارد چون فهم لازم را دارد پس میتواند یک جامعه را اداره کند در تمام این مراحل ایراد منطقی وجود دارد شاید ایشون کل عمر کتاب 
	داستان خونده، چون فهم لازم در بعضی از امور را دارد به این معنی نیست که میتواند جامعه را اداره کند ارسطو این بحث هارا درکتاب منطقش
	ریتادیکس اینهارا لیست کرد
	
	\پاراگراف{ذهن و زبان} ما معتقد بودیم که ذهن ما زبان مارو میسازه ولی ویتگنشتاین اومد گفت نه زبان هست که ذهن مارو میسازه(بحث طولانیه) 
	پس اگر خیلی مغالطات رو تکرار کنیم ذهن ما همش میشه مغالطات ولی در مطلب فلسفی اگر یک مغالطه هم بکنیم مطلب بیهوده میشه چون پازل فکری 
	بهم میریزه مگر اینکه اون گوشه های مطلب باشه
	
	\پاراگراف{فلسفه زبان و واژه شناسی} مدیون نیچه هستیم و بعد سوفستایی ها (پیشا سقراطی ها) که به همه چیز در جهان زبان نگاه میکردند پس
	واژه ها معانی متفاوتی میتوانند داشته باشند مثلا اخلاق برای ما در زبان فارسی فرق داره با واژه مثل اخلاق در زبان عربی برای یک عرب زبان
	چون زبان یک تاریخ داره یک سنت داره و طبق این سنت یک مفهوم تاریخی هم براش ساخته شده خیلی وقتا ما اینارو به عنوان غلط مصطلح استفاده 
	میکنیم مثلا عاقبت گرگزاده گرگ شود گرچه با ادم بزرگ شود یا تربیت نا اهل را چون گردکان بر گنبد است یعنی کسی که بده ذاتش بده یک نگاه
	ذاتگرایانه و اشتباه غلط  که کل روانشناسی میبره زیر سوال اقا شما حاصل بچگی و تحریکات و نیازهای خودتی
	\پاراگراف{مغالطه علت اشتباه } یعنی ما علت اشتباهی رو به یک پدیده نسبت بدیم فلانی خوبه چون خانوادش خوبن هنرمنده چون خانوادش هنرمندن
	چه ربطی داره خانواده میتواند یک عامل مهم باشد ولی تعیین کننده نیست ما دنبال علل مختلف باید باشیم
	
	\زیرقسمت{مغالطه مصادره به مطلوب \متن‌لاتین{Begging the Question}} کسی یک ادعا میکند برای اثبات ادعا یک سری دلایل میارند که در اون دلایل
	انگار از قبل ادعا به عنوان اصل پذیرفته شده 
	\پاراگراف{مثال}
	از کجا میدونی خدا وجود داره\\
	در کتاب مقدس گفته\\
	از کجا میدونی کتاب مقدس درست میگه؟\\
	چون خدا فرستاده\\
	همه عروسک ایرون من رو دارن میخرن\\
	چرا؟\\
	چون پرفروشترین عروسک ساله:|\\
	شک فلسفی دکارت: نمیتونیم مطمئن باشیم خوابیم یا بیدار مثل فیلم ماتریکس شما میتونی مغزی باشی در خمره :)\ \\
	ولی یکی میاد میگه من دستمو دارم میبینم دیگه من دارم میبینم این جهان واقعیه درواقع دکارت میخواد ارتباط بین احساس رو از بین ببره پس 
	نمیتوانیم با احساس احساس رو ثابت کنیم
	
	\زیرقسمت{مغالطه حمله به شخص \متن‌لاتین{Ad Hominem}} بجای اینکه به فکر به دلیل اشاره بشه به شخص اشاره میشه
	\پاراگراف{مثال} اقا تو میگی عراقی شاعر بزرگی بوده همه میدونن این ادم مشکل اخلاقی داشته، هنر را باید فارق از شخصیت هنرمند نگاه کرد\\
	اقا چی میگی هی مصدق مصدق جلال‌الدین فارسی گفته این به ما خمس و ذکات میداده اینکه چون تو مشکل شخصی با خمس و ذکات و اسلام داری دلیل 
	نمیشه مصدق ادم بدی باشه \\ اقا تو چی هی میگی اختلاف طبقاتی تو که حرف کمنیست ورشکسته هارو میزنی توهم از اونایی ارهههه ؟؟ :)))\\
	\زیرزیرقسمت{انواع مختلف} مثلا شخص معروفی یک حرفی زده:‌ اقا شما بی خدا ها چی میگید مگه نیوتن نمیشناسی این خدا رو قبول داشته پس چی میگی\\
	یا مثلا هیچ کس از متفکرین نفهمیدن تو فهمیدی؟\\
	
	\پاراگراف{ببین چه میگوید نه اینکه که میگوید} اگه یکی میگه سیگار کشیدن بده ببین چرا میگه بده نگو خودتم که سیگار میکشی ممکنه یک فرد 
	رو به ریاکاری متهم کنیم مثلا هیتلر اگه بگه ازادی
	
	
	
	
	
	\قسمت{اندیشه نقادانه (شاهین و وریا) قسمت ۲}
		\پاراگراف{نکته} فلسفه عین زندگی فلسفه مربوط به یک جهان خارج از زندگی نیست اگر صحبت از مفاهیم میشه اگر صحبت از ایده میشه ما 
		این هارو در زندگی به کار میبریم از رابطه زن و شوهر تا پدر و مادر باید فلسفی برخورد بشه\\
		فیلسوفان پراگماتیسم امریکایی(عملگرایانه): جان دیوی رورتی ولیام جیمز هیلاری پاتلرم و ....\\
		جان دیووی خیلی به اموزش و پروش اهمیت میداد و میگفت دانش‌اموزان باید روش کشف را یادبگیرند و درگیر این جهان و پروسه کشف و شهود باشند
		\پاراگراف{تعریف جان دیووی از اندیشه انعکاسی} برسی فعال ماندگار و دقیق یک باور یا دانش بر سایه ی استدلاهایی که ان را توجیه میکند
		و نتایجی که می‌توان از ان گرفت\\
		\شروع{شمارش}
			\فقره فعال active: به این معنی که هرکس هرچی به ما گفت نپذیریم خودمان در پرسه فکر کردن شرکت کنیم و در مقابل مجهول passive
			\فقره ماندگار و دقیق: در مقابل ذهنیت غیر انعکاسی یعنی نتیجه گیری سریع
		\پایان{شمارش}
		
	
		\پاراگراف{تعریف ادوود گلیسر} اندیشه انتقادی اندیشه عقلانی و مستدل (ریزن ابل) و انعکاسی که تمرکزش بر این باشه که به چه چیز هایی 
		باور داشته باشیم و چه اعمالی رو انجام بدهیم 
	
		\پاراگراف{رورتی} دموکراسی اولویت داره به فلسفه و مفاهیم وقتی شما تجربه زیستی نداشته باشه مشکل پیشمیاد برات
		
		\زیرقسمت{استدلال}
		\پاراگراف{تعریف} ان چه در یک مطلب فلسفی میخونید مجموعه ای از استدلال هاست که یک سری گزاره هست که احتمالا پیشفرض هست بعد از ان ها 
		یک سری نتیجه که این نتیجه ها در طول یک سری استدلال هستند و ممکن است نتیجه استدلال قبلی پیش فرض استدلال بعدی باشد و مطالب فلسفی 
		معمولا این جوری پله پله میرن جلو و حالا هرکدام از این استدلال ها غلط باشد احتمال اینکه مطلب را کلا نقض بکند خیلی زیاد است بعضی 
		وقتا هم نه یک استدلال اون گوشه های مطلب هست اگر صادق نباشه امکان داره مطلب رو از حیث اعتبار ساقط نکند 
		\پاراگراف{استدلال استنتاجی} (deductive) که یک مدل معروفش مقابسه قیاسی ارسطو هست(syllogism) در این مدل استدلا یک سری 
		مفروضات اولیه هست این مفروضات اولیه باید صدقش اول ثابت بشه بعد اگر استدلال درست باشد به یک سری نتایج میرسیم که مهم نیست چی باشه 
		هرچی باشه درست هست و باید قبولش کنیم
		دو نمونه استنتاج داریم استاج بهترین تبیین (\متن‌لاتین{abductive reasoning}) و یکی دیگر هست \متن‌لاتین{deductive reasoning} 
		
		\پاراگراف{استدلال استقرایی} مدل خیلی معروفی هست هم در ریاضات هم در علم استفاده میشود در ریاضیات یک روش بسیار دقیق هست برای 
		اثبات یک قضیه ریاضی در سری ها خیلی استفاده میشه ولی در عمل و در زندگی روزمره استدلالی است که بر پایه تجربه هست مثلا میگیم قضیه
		ای یکبار اتفاق افتاد یک سری پیامد ها داشت دوباره اتفاق افتاد دوباره همان پیامد هارا داشت دفعه سوم باز هم اتفاق افتاد و باز هم
		همان پیامد هارا داشت پس میگیم حتما درست هست و روش یک مدل میسازیم و در واقع روی یک مجموعه داده (یا تجربه های یکسان از جهان) ما 
		یک نتیجه گیری میکنیم که در این نتیجه گیری هم ممکن هست اشتباه وجود داشته باشد مثال جالبش جاذبه هست بچه از شکم مادر و زمانی که 
		بدنیا میاد جاذبه رو تجربه میکنه همینطوری اروم اروم میفهمه که یک چیزی بسمت زمین بقیه چیز هارو داره میکشه
		
		مهم این هست که از یک تجربه که تکرار میشه ما چطور روش مدل میسازیم مثلا همین مثال جاذبه مثلا مدل ارسطو این بود که همه چیز که زمین 
		هست هر چیزیم ول کنیم که میخواد برگرده به زمین ما هم که از خاک امدیم و به خاک میریم خورشید و ستاره و سیاره ها هم که دور زمین 
		میچرخند پس همه چیز زمین هست هرکه را دور ماند از اصل خویش  بازجوید روزگار وصل خویش
		
		حالا یک کسی مثل نیوتون میاد میگه نه خیر من ازمایشات بیشتری کردم و به این نتیجه رسیدم فقط زمین نیست که جاذبه داره هر جرمی که جرم 
		دارد از ماده تشکیل شده جرم دیگری راکه که ازماده تشکیل شده بایک نیرویی دیگری که بهش میگن قانون جاذبه نیوتون (عدد جی به 
		روی ام۱ ام۲ به روی فاصله به توان دو) این مدل خیلی بهتراز مدل ارسطو بود خیلی بهتر کار میکرد احتمالا خیلی بیشتر به حقیقت نزدیک بود
		اما نسبیت انیشتین گفت نه فضا داره خم میشه و این خم شدن فضاس که شمارا هول میده به سمت اون خطوط خم شدن فضا که بری به سمت جرم 
		بزرگتر
		
		\پاراگراف{الگوریتم} یکی از چیز هایی که بهش کمک میکه بحث پیشامد ها تصادفات احتمالات در برخورد چیزهای مختلف با یکدیگر هست همین را
		ما الگو سازی میکنیم در جامعه شهری در روابط باهم دیگه یعنی مثلا برخورد ما با همجنسگرایان متفاوت هست با برخورد افراد در جامعه غربی
		تجربه زیستی در هر مکان و جامعه یک سری امکان ها به افراد میدهد حالا اون تجربه های زیستی ناشی از اون برخورد های احتمالی هست ناشی
		از مورد هایی که هست که به عنوان پیشامد با ما برخورد میکنند یعنی مثلا من چند بار با چند همجنسگرا برخورد داشتم و بر اساس این 
		برخورد ها چه تجربه زیستی داشتم و حالا چه قضاوتی دارم چه استدلالی دارم این مثال خوبیه برای این که بفهمیم چطوری از طریق استقرا ما 
		قوانین اخلاقی میذاریم مثلا اینکه چرا همجنسگرایی در فرهنگ های گذشته بد بوده احتمالا با تکامل داشته یعنی انسان میخواسته یک نسل جدیدی
		داشته باشه خانواده یه ارگان خیلی مهم بوده و شکل دادن خانواده و نسل بعد ایجاد کردن اگر یک نگاه تکاملی داشته باشیم میتوان فهمید 
		چرا انسان بدوی بد میدونسته همجنسگرایی رو و چرا انسان مدرن دیگه موضوعیت نداره براش اون جهان
		
		\پاراگراف{مغاله مسیحیان در رابطه با همجنسگرایی} ازشون میپرسین اقا انسان ازاد هست که خودش عمل جنسیش رو انتخاب بکنه؟ میگن پس بچه باز
		ها هم نمیشه ایرادی بهشون گرفت خوب این چه ربطی داشت این قیاس مع‌الفارق هست این سمت دوتا ادم بالغ هستند اونور یک بچه هست چه ربطی داره
		در واقع اینجا بخاطر مدلی که ذهن کار میکنه و میخواد اینو غیراخلاقی جلوه بده این دوتا رو به هم ربط میده که این هم بخشیش برمیگرده 
		به استقرا 
		پاراگراف{سوال} برای همین ما در برخورد با تاریخ نمیتونه از سر احساس باشه ما ذهن امروزی خودمون رو باید بذاریم توی اون کانتکس مثلا
		انسان امروزی قرن ۲۰ ۲۱ میتونه راجع به پایین بودن سن ازدواج در ۲ هزار سال پیش صحبت بکنه یا با دید امروز در مورد جنگ در گذشته صحبت
		بکنه چرا؟ وقتی با استدلال های استقرایی ما جلو میریم اولا بحث احتمالات هست و در ساحت ریاضی اتفاق می افته و ما با سود کارکرد و نتیجه
		روبرو هستیم مثلا میگن معتادانی که از سرنگ مشترک استفاده نمیکنند خطر مرگ کمتری دارند بعد دولت بین معتادان سرنگ پخش میکنه یا مثلا یک
		تحقیقی میگفت که مردانی که ختنه شده هستند عمر بیشتری دارند(البته این رد شده این مغالطه یهودیان و مسلمانان است) و یک چیز دیگه ما
		وقتی راجع به استقرا داریم صحبت میکنیم این امادگی باید داشته باشیم که این استدلال به زودی نابود بشه یا رد بشه اینجا باید یک مرزبین
		این نمونه از استدلال های استقرایی با نسبی گرایی بذاریم مثلا وقتی میگیم زن ایرانی حقوق متفاوت داره مثلا با زن امریکایی و میخواهیم 
		ظلم وارد به زن ایرانی رو اینجوری توجیه کنیم چرا چون به نسبت جغرافیا فرهنگ و جا میخوای بگی حقوق متفاوتی وجود دارد درواقعه داری
		یک سری شواهد رو نادیده میگیری  و مغالطه میکنی
		
		\پاراگراف{در مورد دموکراسی} شاهین میگه این عقیده را دارد که بدون بوجود اوردن امکانات دموکراتیک در جامعه این ناممکن هست که بتوانیم
		فکر دموکراتیک را در اون جامعه نشر بدیم یا پخشش کنیم ترویج بدیم در واقع در استدلالات استقرایی ما باید این مرز را در نظر بگیریم ما 
		قرار نیست توجیه کننده باشیم بحث اینه که بر اساس استدلال امار و احتمالات چقدر امکان سود و چقدر امکان اینحه به نتیجه نزدیک بشیم هست
		و چقدر کارکد داره این قضیه
	
	
		\پاراگراف{تفاوت استنتاج و استنتاج بهترین تبیین}  مثال استنتاج عادی انسان میرا است سقراط انسان است پس سقراط میرا است این استدلال
		اگر مفروضات رو قبول بکنیم نتیجه گارانتی شده و ۱۰۰ در صد هست اما سارا مهمونی نمیاد چون خجالتی هست و جمع اذیتش میکنه این موجه میکنه
		باور مارو ولی ۱۰۰ در ۱۰۰ نمیکنه به این مدل میگن بهترین تبیین
		
		\پاراگراف{مفروضات، بدنه استدلال و نتایج در استدلال استنتاجی } در استنتاج اگر مفروضات غلط باشه و استدلال درست باشه ۱۰۰ در ۱۰۰ نتیجه 
		غلط خواهد بود مگر اینکه استدلال غلط باشه که از فرض غلط نتیجه درست بگیری البته منطق یکی دوتا نیست ما کوانتوم لاجیک فرمال لاجیک 
		مدال لاجیک و ... داریم در بعضی هاش این گفته نقض میشه 
		
		\پاراگراف{مثال در زندگی اجتماعی}  چرا در یک سری کشور ها روز یکشنبه مردم سروصدا میکنند اما در یک کشور دیگه سکوت مطلق هست؟ چون
		اون جامعه امکانات و احتمالاتی که سروصدا رو توجیه بکنه ندارند مثلا المانی ها خونه هاشون زیاد باغ نداره اما توی یک کشور دیگه که خونه
		باغ دارند باغبانی میتونه توجیه کنه سروصدا رو این برمیگرده به سود کارکرد و نتیجه نه اخلاق
		
		\پاراگراف{مثال} اگر در انتخابات تقلب بشه مردم اعتراض میکنند مردم اعتراض کردند پس در انتخابات تقلب شده این غلطه چون برعکسش صادق 
		نیست یعنی مردم در انتخابات اعتراض نکردند پس یعنی تقلب نشده شاید مردم ندانند که انتخابات مشکل داشته استدلال درسته ولی مفروضه غلطه
		
		\پاراگراف{مثال} کسانی که به تکامل باور دارند فکر میکنند انسان از نسل میمون است(این خودش غلطه تکامل میگه ما و میمون ها از نسل 
		ایپس ها هستیم که اونا منقرض شدند در واقع باید بگیم ما و میمون ها از نسل امیپ ها هستیم)، احسان به تکامل باور دارند پس احسان از 
		نسل میمون است.
		
		این اشتباه هست اولا که مفروضه غلط هست و دوم اینکه اگر کسی به چیزی باور داشته باشد دلیل نمیشه که ممکنه باورش غلط باشه ممکنم هست 
		درست باشه و اگر درست باشه نمیشه گفت فقط خودش از نسل میمون هست که کل جهان از نظر باور اون از نسل میمون هستند
		
		
		\پاراگراف{مثال} همه آمریکایی ها احمق هستند چامسکی آمریکایی است پس چامسکی احمق هست.
		
		حتی اگر ما بر این باور باشیم که چامسکی احمق هست نمیتوان نتیجه گرفت همه امریکایی ها احمق هستند در واقع این جنرالایز کردن تعمیم یک
		بخش به همه و کلی گویی است و مفروضه هم غلط هست همه امریکایی ها احمق هستند یک مفروضه غلط هست
		
		\پاراگراف{مثال} بیانسه در پاریس بدنیا امده، هرکه در پاریس بدنیا بیاید پنیر دوست دارد  پس بیانسه پنیر دوست دارد 
		
		هر دو مفروضه اشتباه استدلال اشتباه ولی نتیجه درست
		
		\پاراگراف{مثال} اندیشه نقادانه به ما یاد میدهد چگونه قضاوت بکنیم چگونه سره را از ناسره تشخیص دهیم
		
		\پاراگراف{استدلال صادق Sound یا Valid} به استدلالی گفته میشود که هم مفروضات هم بدنه استدلال درست باشد بهش معتبر هم میگن در این حالت 
		دیگه مهم نیست نتیجه چی هست هرچه هست باید بپذیریمش این درواقع فرمال لاجیک هست
		
		\پاراگراف{مثال} اخوان ثالث در مشهد بدنیا امده مشهد در ایران است اخوان ثالث ایرانی است این مدل استدلال قیاسی ارسطو هست،‌از جز داریم
		به کل میرسیم یا کل رو توضیح میدیم
		
		\پاراگراف{نکته مهم} اول باید به مفروضات و بعد به بدنه نگاه کرد بعد به نتیجه بعضی وقتا ادما میگن فرض بگیریم که فلان باید حواسمان
		باشد که ایا ما این فرض رو قبول داریم یا نه اگر نه سرش بحث کنیم و اگر اره بعد باید بدنه استدلال چک شود اگر هردو رو قبول کردیم باید
		نتیجه را قبول کنیم 
		\پاراگراف{مثال} فرض کنیم که جهان خالقی دارد به نظر شما این درست نمیاد اول باید ما ثابت بکنیم پاسیبیلیتی چک بشه پروبابلیتی با 
		پاسیبلیتی فرق میکنه اول باید ثابت بشه که میشود همچین فرضی کرد بعد نتیجه گرفت تفاوت امکان و احتمال مهم هست درواقع بحث این هست 
		که چگونه بر اساس احتمالات ما امکانات رو بوجود بیاریم و ان ها را بفهمیم در اخر مفروضات خیلی مهم است
		
		\پاراگراف{مثال خیلی جالب} هر معلول علتی دارد (خوب این قبول) این جهان معلول است پس این جهان علتی دارد اینجا باید ثابت بشه که این 
		جهان معلول است این بازی زبانی است این مصادره به مطلوب است در واقع در بحث علت و معلول وقتی میخوان مغلطه بکنند علت رو به عنوان 
		امر پیشین بهش اشاره میکنند در صورتی که علت یا در اینده هست یا حال استمراری من ورزش میکنم چون میخوام سلامت باشم یا الان سلامت هستم
		میخواهم نگهشدارم یا در اینده میخوام سلامت بشم و الان سلامت نیستم
		
		\پاراگراف{مغلطه مرد کچل} میگن احمد اقا زلف پر پشتی داره اگر یک تار ما بکنیم کچل میشه؟ نه نمیشه دوتا چی کچل میشه؟ نه نمیشه ما با
		این استدلال به جایی میرسیم که اگر همه موهای یارو هم بکنیم یارو کچل نمیشه در واقع واقعیتش این هست که بله احمد اقا اندازه همون 
		یک تار مو کچل میشه این مغالطه اختلاسگران هست میگن اقا حالا این همه دزدیدن بردن حالا اینی که من بردم که چیزی نیست مشکل این مغلطه این 
		هست که ما اینجا نمیتونیم مرزی برای کچلی تعریف بکنیم ما نمیتونیم بفهمیم کچلی از کجا شروع میشه و از کجا تمام میشه 
		
		\پاراگراف{مغلطه معنایی} این مغالطه هم وکلا زیاد استفاده میکنند در اروپا شما یک دلار هم بدزدی به عنوان دزد میگیرن میبرنت در اروپا 
		سال ۲۰۰۷ یک سری ادم باعث ورشکستی بانک ها شدن یارو واسه خودش ماهی ۱۰ میلیون پوند حقوق نوشته بود از پول ملت تو بانک بعد که یارو رو 
		گرفتن وکیلش میگفت نه اقا دزدی چیه ایشون \متن‌لاتین{inevating accounting} کرده یعنی خلاقیت در حساب داری یعنی اسم دزدی رو تغییر داده 
		بود مثلا به سرباز های نیروی هوایی نمیگن زن و بچه ملت بزن که میگن \متن‌لاتین{aim at the target} یعنی هدف رو بزن وقتی ادم میشه هدف
		زدنشم اسونه یا مثلا یارو حرف از ازادی زده میگه تشویش اذهان عمومی کرده یا مثلا دزدی سر گردنه اسلام اسمشو کرده غزوه و میگه خوبه اسم
		مال ملتم میذارن غنیمت یا مثلا خمس مال مردم خوری یا مثلا اسم دروغ شده تقیه 
		\پاراگراف{نکته} سفسطه رو به عنوان یک مغلطه تعریف کردن که این خودش یک مغلطه هست اگر سفسطه رو به عنوان یک یک روش در فلسفه بپذیریم
		دیگه مغلطه نیست چون دیگه با زبان طرفیم یا مثلا تعزیر یا کافر یا مشرک
		
		\پاراگراف{مغالطه مسئولیت اثبات \متن‌لاتین{burden of proof}} مسئولیت اثبات ادعا بر دوش مدعی است یعنی وقتی داری استدلال میکنی باید 
		مسئولیت اون استدلال رو بپذیری چه در حمله چه در دفاع
		\پاراگراف{مثال} مثلا اگر ادعا میکنی ماه به دور زمین میگردد باید ادعا ثابت کنی مثال بامزش جوک ملانصرادین هست که میگن چوبش رو میکنه 
		تو زمین میگه اینجا وسط زمینه بهش میگن از کجا میدونی میگه برو متر کن 
		
	\قسمت{استدلال استقرایی}
	\پاراگراف{سوال} از کجا تشخیص دهیم که استدلال استقرایی یا استنتاجی هست و حالا فهم این موضوع قرار چه کمکی به زندگی واقعی ما بکنه؟
	\زیرقسمت{پیشنی و پسینی \متن‌لاتین{a priori and a posteriori}} 
		یعنی یک گزاره نیاز به تجربه دارد یا ندارد مثلا همه مردان مجرد ازدواج نکرده اند نیاز به تجربه ندارد یا ۲+۲=۴ یا مثلث ۳ ضلع دارد
		اینا پیشینی هستند و اثباتشون رو میشه در یک عالم ذهنی برسی کرد استنتاج این مدلی هست و نیازی به تجربه جهان نیست مگر برای اثبات 
		پیش فرض ها ولی استقرار پسینی هست در ریاضیات از هردو استفاده میشه ولی مدل استقرا ریاضی با استقرا علم و تجربه یکم فرق داره مثلا
		اینکه هر عدد زوج مجموع دو عدد اول است خوب تا کی میتونیم بریم جلو برای n  و n+1 اگر اثبات بشه دیگه قبولش میکنیم در واقع در علم
		استقرا پیشینی نیست پسینی هست مثلا میگیم هروقت صدای ریل میاد یعنی داره مترو میاد ولی خوب شاید چنتا کارگر داشتن کار میکردند اونجا
		(صدا در فلز زود تر حرکت میکنه)
		
		\زیرقسمت{چطور باید باید با استقرا  در علم برخورد کرد؟}
		مثلا قضیه فیساغورس با استنتاج ثبات میشه اما نسبیت انیشتین که استقرا هست و شاید ما امکان تجربش رو نداریم چطور باید ثابتش کرد؟
		از تریق مشاهده مثلا یک ذرات ریزی که طول عمرشون در محیط مادی کسری از ثانیه است اما ما ۳ ۴ ثانیه میتونیم ببینیمش میگیم این دلیلش
		نسبیت انیشتین هست اینقدر سرعت بالاست که زمان کش میاد یا مثلا زمان خورشید گرفتگی چون نور اطراف خورشید خم شده بود ما میتوانستیم
		یک ستاره ای که در حالت عادی نمیدیدمش را ببینیم
		
		\زیرقسمت{مشکل استقرا} هیوم به این نوع استدلال تاخته بود و معتقد بود استدلال خوبی نیست چون اینکه چندبار ما چیزی رو ببینم دلیل بر
		درست بودنش نیست مثال حیوان و صاحبش که بهش غذا میدهد از راسل. 
		
		هیوم میگه فقط احتمال بهش میده اثباتش نمیکنه
		
		\پاراگراف{خوب حال پس علم زیر سوال میره؟} نه چون تمام چیزی که داریم تجربه علمیست راه دیگری نداریم از اجداد ما تجربه علمی بوده تا
		الان یعنی چند بار سنگ زدن بهم دیدن ع جرقه میزنه دیده از دستش میتونه استفاده کنده یا زمانی که روی دو پا حرکت کرد یا زمانی که مثلا
		عقلانی میشه ۱۰ بار انقلاب شده به این دلایل پس احتمالا بار ۱۱ هم به این دلایل انقلاب میشه (احتماالا)
		
		\پاراگراف{نکته برای حمله به شخص} این که فرد رو به نادانی متهم میکنه دلیل خوبی نیست خوب کسی که همه موضوعات رو بلد نیست باید اگر
		ایرادی هست به استدلالش گرفته بشه مثلا راسل در کتاب فلسفه غربش از همه فلاسفه عذرخواهی میکنه چون میگه هیچ کس که اشراف کامل نداره پس 
		حتما ایراداتی هست
		
		\زیرقسمت{شروط لازم و کافی \متن‌لاتین{sufficient and necessary}}
		در ریاضات زیاد هست شرط لازم و کافی یا اگر و فقط اگر مثلا بخوایم بگیم امتحان دادن شرط لازم است برای قبول شدن ولی شرط کافی نیست یا مثلا
		تلاش کردن لازم است ولی کافی نیست برای موفقیت یا مثلا میگن شواهد کافی وجود نداشت برای متهم کردن لازم بودن به تنهایی کافی نیست یعنی این
		که یک نفر کشته شده و ما فقط یک متهم داریم خوب متهم داشتن لازم است برای پیدا کردن قاتل ولی کافی نیست شاید قاتل کس دیگری است و ما 
		نمیدونیم
		\پاراگراف{برخورد به صورت استنتاجی و استقرایی با شروط} ما با ملزومات استنتاجی رفتار میکنیم ما نمیتوانیم بفهمیم که انسانی زنده نباشد
		اما راه برود اما بر خورد با انچیزی که قرار است کمک بکنه به کفایت استقرایی هست مثلا زنده بودن لازم است برای راه رفتن اما کافی نیست
		
		\پاراگراف{حدود شروط چگونه است } کی مثلا میتوانیم بگیم دیگه کافی است؛ بعضی وقتا مجموع یک سری شروط لازم باهم کافی میشوند مثلا مثال 
		باران امدن برای باران ما سیاره میخوایم ابر میخوایم اب میخوایم و .... مثلا در هندسه میگیم کی مربع داریم زمانی که ۴ تا خط صاف داشته
		باشیم بین خطوط زاویه قائمه باشد و هم اندازه باشند خطوط همه این شروط لازم باهم شرط کافی میشوند در واقع چون مشخص کردن حدود شروط کافی 
		سخت است ما شروط رو بر اساس کارد و سودشون در نظر میگیریم
		\پاراگراف{نکته} بعضی وقتا نه لازم است نه کافی، هم لازم است هم کافی، لازم است اما کافی نیست، کافی است اما لازم نیست مثلا خدا کافی است 
		ولی لازم نیست مثلا در ۲ ثانیه اول بیگ بنگ
		
		\پاراگراف{۴ علت ارسطو} علل اربعه یا ۴ علت ارسطو علت فاعلی، علت مادی، علت صوری، علت غایی
		
		علت و العلل هم برای خودش لازم و کافی هست هم معلولش(؟) 
		
		شروط لازم کافی نستند مثلا مثال گچ و بچه ها یا مثال نامه برای جاسوسی، شرط لازم صلح قرارداد است اما قرار نیست که هر قراردادی یعنی اینکه
		صلحی وجود دارد
		
		\زیرقسمت{مغالتات این قسمت}
		\زیرزیرقسمت{پهلوان پنبه یا مرد پوشالی \متن‌لاتین{straw man}}
		 در واقع به گزاره ای حمله میشه که با اون نقص و صحت یک گزاره ثابت نمیشه انگار یک گزاره از 
		بدنه یک استدلال میاری بیرون و داری به اون حمله میکنی بعضی وقتا اصلا یک تیکه از حرف رو میگیرن تغییرش میدهند و بعد به همون تیکه حمله
		میکنند مثلا میگن گاز گلخانه ای بد است و باید کاهش داد اینارو بعد میان میگن یعنی شما میگی ما نباید سوار ماشین بشیم مثلا در تکامل 
		میگن مگه نمیگی ما از نسل میمونیم چرا پس میمون هنوز وجود داره (خب کی تکامل گفت ما از نسل میمونیم)
		
		\زیرزیرقسمت{مغالطه ابهام ambiguity} استفاده از واژگانی که به کیفیت اشاره میکنند و نسبی هستند مثل دور نزدیک کم زیاد باید تعریف کرد
		مثلا میگن در قران گفته والعرض دهاها این یعنی گستردن یک معنی دورش تخم مرغ هست یعنی میخواد بگه زمین گرده چه ربطی داشت اصلا اصلا اگر 
		منظورمش اون باشه زمین که تخم مرغی نیست زمین گرده و یا میگه ندیدی که خدا شب را در روز و روز را در شب میکند(سوره لقمان) خوب این چه 
		ربطی به گرد بودن زمین داره
		
		\زیرزیرقسمت{کنه و وجه} یعنی یکی از صفات یک چیز رو بگیریم و ان رو به همش گسترش بدیم مثلا انسان چیزی جز میمون ناطق نیست یا مثلا دین 
		چیزی نیست جز روش رسیدن به خداوند یا مثلا دموکراسی یعنی ازادی میگن اقا دموکراسی که فقط ازادی نیست ازادی یکی از نتایج دموکراسی هست 
		برای مقابله دقیقا باید دست بذاریم روی جایی که طرف میخواد هایلایتش کنه یا مثلا میگن چپ چیه چپ یعنی کمونیست یعنی همه چی اشتراکی اینا
		با مادر و خواهر خودشون زنا میکنند
	\قسمت{مغالطه در استدلال(۳) مغالطات ربطی}
	\زیرقسمت{مغالطه مصادره‌به‌مطلوب}
	نام لاتین \شروع{لاتین} begging the question\پایان{لاتین}
	نتیجه استدلال که باید اثبات شود،‌ در فرض اثبات شده در نظر گرفته شود.\\
	
	
\end{document}