\documentclass{article}
\begin{document}
	\title{My Git Notes}
	\maketitle
	\newpage
	\tableofcontents
	\newpage
	\section{Branches}
	
		branches are separate isolate workspace to prevent making main line of work unstable.
		unlike subversion which copy entire working directory git just creating a new pointer to creating a new branch. 
		so master branch is just a pointer to the last commit on the main branch.
		HEAD pointer used for pointing to the current branch.
	
		\begin{tabular}{l l}
			git  branch branch-name          & create new branch              \\
			git  branch                      & list of branches               \\
			git  status                      & find out the current branch    \\
			git  switch branch-name          & switch the branch-name         \\
			git  branch -m old-name new-name & renaming branch                \\
			git  log --oneline --all         & see commit across all branches \\
			git  branch -d branch-name       & deleting branch                \\
			git  branch -D branch-name       & force deleting branch          \\
			git switch -C branch-name        & create and change branch       \\
			git branch --merged              & list of merged branches        \\
			git branch --no-merged           & list of no merged branches
		\end{tabular}

		if didn't merge the branch deleting may occur a error
		
	\subsection{Comparing Branches}
		\begin{tabular}{l l}
			git log master..bugfix                      & show all commits in the bugfix                            \\
			git diff master..bugfix                     & show all the changes                                      \\
			git diff bugfix                             & show differences between current branch and bugfix branch \\
			git diff [--name-only --name-status] bugfix & show differences between current branch and bugfix branch
		\end{tabular}
	
	\subsection{Stash}
		if we have some changes in some branch and didn't commit yet changes could get lost by changing branch because when changing branch
		git reset our working directory to the last snapshot that saved on the latest commit on the target branch.\\
		stashing something means store it on the safe place.\\
		by default new untrack files are not included in the stash \\
		each stash has a unique identifier.\\
		\begin{tabular}{l l}
			git stash push -m 'massage'  & create new stash             \\
			git stash push -am 'massage' & to contain new untrack files \\
			git stash list               & list of stash                \\
			git stash show stash-id      & show the changes             \\
			git stash apply stash-id     & apply the stash              \\
			git stash drop stash-id      & delete the stash             \\
			git stash clear              & remove all the stash
		\end{tabular}
	
	\subsection{Merging}
		bring up changes from a branch to another branch.
		use --graph option with git command its a better representation of git branches and diverge.
		
		\subsubsection{Fast Forward Merge}
			when the main line of work didn't have diverge and we have a direct linear path between two branches.
			disabling fast forward merging may have some benefits (reverting the commits are easier)\\
			\begin{tabular}{l l}
				git merge branch-name         & merge branch-name to the current branch                             \\
				git merge --no-ff branch-name & merge branch-name to the current branch with no fast forward option \\
				git config --global ff no     & disable fast forward global                                         \\
				git config ff no              & disable fast forward on the current repository
			\end{tabular}
		\subsubsection{3-Way Merge}
			when the main line of work have diverge git look the last snapshot of two branches and make merge commit and merge the changes.
			
			
		\subsubsection{Conflict}
			happens when the same line of code changes in the different branches, or delete the file in another branch or add the same file
			with different contents in two branches\\
			with the git status command see the file(s) that cause the problem\\
			open files and solve the problem (REMEMBER DO NOT ADD NEW LINE OF CODE)\\
			add the file to the staging area then use git commit message.
			kdiff and p4merge are cross platform graphical tools to resolving the conflict.\\
			git config --global merge.tool p4merge\\
			git config --global mergetool.p4merge.path "path/to/the/p4merge"\\
			git merge --abort back to the state before we stated the merge\\
		\subsubsection{Undoing Faulty Merge}
			two ways:
			\begin{description}
				\item [rewriting the history(deleting the merge commit)]
					both the master and the head point to the last commit which is the merge commit we should move both of them to point to the commit before creating branch by using git reset command, the merge commit determine garbage by the git and one in a while remove by the git\\
					git reset --hard HEAD$\sim$1\\

					\begin{description}
						\item reset command options 
						\item[soft] effect on last snapshot
						\item[mixed] effect on last snapshot and staging area
						\item[hard] effect on last snapshot, staging area and working directory
					\end{description}
				\item [reverting the merge] the last commit have two parents one in the master branch(1) and one in the bugfix branch(2) so we should use the revert command like this git revert -m 1 HEAD
			\end{description}
			the first approach is ok if this history is local in our local repository but its not the best practice if we are working on the remote repository in this situation we should use the second approach we creating a new commit and cancel all changes on this commit.
		\subsubsection{Squash Merging}
			if we don't have good commits on the branches we combines all the changes in the branch and then merge it to the master branch 
			so we have clean linear history path this is what we call squash merging this new commit not a merge commit because it doesn't have two parent commit. we should use for small and short lives branches like bug fixes of features can implemented in a few hours or a day.\\
			git merge --squash branch-name(add all changes to the staging area ready to commit)\\
			git commit\\
			it's very IMPORTANT TO DELETE THE BRANCH unlike it may occur conflict in the future because git shows it on the unmerged branches so delete it with -D option
			
		\subsubsection{Rebasing}	
			an other technique to bringing changes from a branch to another\\
			rebasing mean change the base commit of a branch e.g we have bug fix branch which master v1.1 is its base commit of the branch and the master branch current version is v.1.2 so we have to use 3 way merge method or change the base commit (rebase) of the current branch to the master v.1.2 and we can use fast forward merge its may sounds a cool idea but rebasing rewrites the history and only on local repository we should use it.\\
			
			Why rebasing is rewrites the history?\\
				what the git actually do is that, git create new commits like the commits on the branch with the base we want then move the branch pointer to point the new commits, git do not change the commit because commits on git are immutable\\
			
			git switch target/branch\\
			git rebase master(tell git to rebase the branch to the last commit of the master)\\
			if rebasing occur conflict we should resolving the conflict like before then use the below command \\
			git rebase --continue\\
			git rebase --skip(skip the current commit and move on the next commit)\\
			git rebase --abort(abort the rebase and take us back to previous state before rebasing)\\
			git config --global mergetool.keepbackup false		
			
		\subsubsection{Chary Picking}
			think of a situation that we're working on a branch and we have interesting commit whit some changes and we want to add this commit to the master but we're not ready to merge the branch so we chary pick the particular topic and apply it on the top of the master\\
			on master: git charypick commit-id\\
			if there's a conflict resolve it then commit the changes.
		
		\subsubsection{Picking Files From Another Branch}
			like chary picking but interested on a single file not whole commit \\
			git restore --source=branch-name --file-name(-- for file name is optional and if git doesn't recognize the file name)
\end{document}
