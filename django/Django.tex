\documentclass{article}
\usepackage{listings}
\begin{document}
	\tableofcontents
	\section{MySQL connection}
	\begin{enumerate}
		\item install mysqlclient
		\item set mysql as database engine
		\item defaults -> OPTION -> "read\_default\_file": "path/to/mysql.cnf"
	\end{enumerate}
	\section{Admin Panel}
	\section{Models}
		\subsection{verbose\_name}
		\subsubsection{Each field type, except for Keys}
		\begin{verbatim}
		first_name = models.CharField("person's first name", 	max_length=30)
		\end{verbatim}
		\subsubsection{for Keys field}
		\begin{verbatim}
			sites = models.ManyToManyField(Site, 	verbose_name="list of sites")
		\end{verbatim}
		we can use this approach for the other field type too.
		\subsubsection{for Model}
			create subclass Meta in the model then add verbose\_name and verbose\_name\_plural attributes.
		\subsection{on\_delete}
			\subsubsection{models.CASCADE}
				 deletes the object containing the ForeignKey.(when parent delete related children delete too)
			
			\subsubsection{models.SET\_NULL}
				Set the ForeignKey null; this is only possible if null is True.(if parent delete children set the foreign key null)
				
			\subsubsection{models.Protected}
				Prevent the deletion, first delete children then delete parent.
	
	\section{Installing an App}
		add 'app\_name' or 'app\_name.apps.MainConfig' to the INSTALLEDAPP in the setting with the second approach we add some configuration
		to the app like verbose\_name
	\section{Farsi language and Tehran Time}
		LANGUAGE\_CODE = 'fa-ir'  TIME\_ZONE = 'Asia/Tehran'
		
	
\end{document}