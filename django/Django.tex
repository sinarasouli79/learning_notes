\documentclass{article}
\usepackage{listings}
\begin{document}
	\tableofcontents
	\section{MySQL connection}
	\begin{enumerate}
		\item install mysqlclient
		\item set mysql as database engine
		\item defaults -> OPTION -> "read\_default\_file": "path/to/mysql.cnf"
	\end{enumerate}
	\section{Admin Panel}
	\section{Models}
		\subsection{verbose\_name}
		\subsubsection{Each field type, except for Keys}
		\begin{verbatim}
		first_name = models.CharField("person's first name", 	max_length=30)
		\end{verbatim}
		\subsubsection{for Keys field}
		\begin{verbatim}
			sites = models.ManyToManyField(Site, 	verbose_name="list of sites")
		\end{verbatim}
		we can use this approach for the other field type too.
		\subsubsection{for Model}
			create subclass Meta in the model then add verbose\_name and verbose\_name\_plural attributes.
		\subsection{on\_delete}
			\subsubsection{models.CASCADE}
				 deletes the object containing the ForeignKey.(when parent delete related children delete too)
			
			\subsubsection{models.SET\_NULL}
				Set the ForeignKey null; this is only possible if null is True.(if parent delete children set the foreign key null)
				
			\subsubsection{models.Protected}
				Prevent the deletion, first delete children then delete parent.
		\subsection{Relationships}
			best practice = name the foreign key field be the name of the model, lowercase
			\subsubsection{One To Many}
				\begin{verbatim}
					one_to_many_field = models.ForeignKey(ParentClass, on_delete)
				\end{verbatim}
				\paragraph{recursive relationships}
					to create recursive relationships; an object that has a one to many relationship with it self do like this
					\begin{verbatim}
						models.ForeignKey('self', on_delete=models.CASCADE)
					\end{verbatim}
				
	\section{Managers}
		models.models.field() here models is a manager
		in models create class for self defined manages which inherit from models.Manage e.g, manager for published articles
	\section{Installing an App}
		add 'app\_name' or 'app\_name.apps.MainConfig' to the INSTALLEDAPP in the setting with the second approach we add some configuration
		to the app like verbose\_name
	\section{Farsi language and Tehran Time}
		LANGUAGE\_CODE = 'fa-ir'  TIME\_ZONE = 'Asia/Tehran'
	
	\section{Views}
		\subsection{generic DetailView}
			\begin{tabular}{ll}
				\hline
				Attributes & \\ \hline
				model & model that want to create form for it \\
				fields & a list off model field to represent in the form\\
				template\_name & form template name 
			\end{tabular}
			\subsubsection{How to send data(e.g, pictures) throw django using HTML forms}
				set encrypt to multipart/form-data in the form html element.
			\subsection{model.get\_absolute\_url}
				absolute url for the model.
				\subsubsection{use it in the templates}
					\{\{instance.get\_absolute\_url\}\}
	\section{Authentication}
	
		admin and auth app in installed apps\\
		field error and non field error
		
		\subsection{Create a Login Page}
			\begin{itemize}
				\item create an app
				\item use django contrib.auth urls
				\item create a template for login page(default django view looks for registration/login.html)
				\item create view and template for login\_redirect\_urls
				\item login\_redirect\_url in settings.py
				\item login required decorators or mixins
			\end{itemize}
			
			\subsubsection{LOGIN\_URL}
				a variable in the django settings file, where the request are redirect when using login\_required decorators or LoginRequired mixin
				
			\subsubsection{LOGIN\_REDIRECT\_URL}
				a variable in the django settings file, where request are redirect after login if LoginView doesn't get a next Get parameter.
		\subsection{login\_required decorator}
			redirect request to the settings.LOGIN\_URL if user isn't logged in.
		
		\subsection{LoginRequired Mixin}
			can achieve the same behavior as with login\_required by using LoginRequired mixin.
	\section{Templates}
		\subsection{Static Files}
			\subsubsection{STATICFILES\_DIRS}
				its a additional location that static\_files app will traverse if the FileSystemFinder is enabled, e.g. if you use the collectstatic or findstatic management command or use the static file serving view\\
				it's default values is an empty list\\
				STATICFILES\_DIRS = [\\
				BASE\_DIR / "static",\\
				'/var/www/static/',\\
				]\\
			
			\subsubsection{include tag}
				use to load and renders it in the current context.
				templates name [].
				pass a value to a template form another template [].
			\subsubsection{load static and static tag}
				\begin{description}
					\item [\{\%load static\%\}] use whenever call a static file
					\item [href = " \{\% static 'url' \%\} " ]
				\end{description}
			
		\subsection{Django Tweak}
			tweak the form field rendering in templates no in python level form definitions, altering CSS classes and html attributes are supported.
		\subsection{Fields Error}
		\subsection{Non Fields Error}
			like wrong password or username
			\subsubsection{Disable JavaScript if Firefox}
			use about:config 
		\subsection{include tag}
		\subsection{block tag}
		\subsection{url tag}
			\{\% url 'main:student\_course\_list' user.get\_username \%\}
			for parameter don't use quote
		\subsection{truncate tag tezmplate filter}
		\subsection{foreign key value in a template}
		
	\section{ORM}
		\section{and, or in query}
		\section{foreign key value in the query}
	\section{Form} 
		\subsection{Raw Html Form}
		\subsection{Pure Django Form}
		\subsection{Django Model Form}
		\subsection{form validation}
		\subsection{how to add styles to a template form}
			\begin{enumerate}
				\item install crispyapp and crispy bootstrap 5
					\subitem read an article in the simple as nothing weblog 
					\subitem update requirements.txt
					\subitem add crispy to installed app 
					\subitem a setting attribute in for bootstrap crispy
				\item use in the template
					\subitem load crispy in the template
					\subitem in the template \{\{form.crispy\}\}
				\item more config
					\subitem costume field position(using as\_crispy\_field)
					\subitem (position of each section)class col6 in simple as nothing doc
				\item add admin lte form style to the form
				
				\item block for title in the base
				\item bug:show tag in the article list
					\subitem use strip and safe tags filter
					
			\end{enumerate}
%		 whats google Books Ngram Viewer 
%	\section{CBV}
%	\section{Dynamic urls}
%	\section{mixins multiple inheritance}
%	
%	TODO:
%	[ ] get context data
%	systemctl --user restart wireplumber
%	
%	font cdn
	
\end{document}