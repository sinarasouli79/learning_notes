\documentclass{article}
\usepackage{graphicx}
\usepackage[localise=on]{xepersian}
\settextfont{XB Yas}
\begin{document}
	\عنوان{آموزش جنگو مهران تعریف}
	\عنوان‌ساز
	\صفحه‌جدید
	\فهرست‌مطالب
	\صفحه‌جدید
	\قسمت{قسمت ۵۱ نوشتن mixin های شخصی برای مدیریت بهتر کد}
		\زیرقسمت{‌\متن‌لاتین{‌Bug Fix}}
		لینک به اسم نویسنده داخل صفحه اصلی زیر مقاله ها درست کار نمیکنه چون داخل ویو کلاس یوزر خود جنگو لود میکنه ولی ما از 
		کلاس یوزر شخصی استفاده میکنیم(داخل تمپلیت ها هم به جای اشتباهی لینک شده بود و خود تمپلیت هم همه مقاله هارا نشان میداد هم 
		منتشر شده هم بایگانی) و استریپ کردن تگ ها داخل صفحه اصلی و صفحه مقالات نویسنده

		\زیرقسمت{فقط سوپر یوزر بتواند فیلد های نویسنده و وضعیت یک مقاله را مشخص کند}
		برای پیاده سازی این قابلیت از میکسین ها استفاده میکنیم هم کد تکراری کم میشه هم نگهداری اسان تر میشه.
		یک فایل \متن‌لاتین{mixins.py} داخل پوشه اپ میسازیم و یک کلاس داخل درست کنیم داخل کلاس متد dispathc را override
		 میکنیم dispatch متدی هست که ریکوست رو میگیره و یک ریسپانس برمیگردونه می خوایم بگیم اگر یوزر ریکوست سوپر یوزر بود
		 فیلدهای عادی، وضعیت و نویسنده را به فرم اضافه کن درغیر این صورت اگر نویسنده بود فقط فیلد های عادی رو اضافه کن و
		 در اخر اگر هیچ یک نبود صفحه را نمایش نده(ارور بده بهش)
		 \begin{figure}[h!]
		 	\includegraphics[width=\linewidth]{mehran-tarif-course-code-pic/FormFieldMixinPic.jpg}
		 \end{figure}
		 میکسین را در فایل \متن‌لاتین{views.py} ایمپورت میکنیم و کلاس مربوط(ساخت مقاله) ازش ارث بری میکنه
		 
		 داخل فایل تمپلیت هم فیلد های نویسنده و وضعیت که بهشون استایل دادیم داخل یک شرط قرار بگیره که تنها وقتی که کاربر
		 سوپر یوزر هست نمایش داده بشه
	
		\زیرقسمت{فیلد نویسنده به صورت پیشفرض برای نویسنده مقدار داشته باشه}
			چون نویسنده نمیتونه فیلد های نویسنده و وضعیت یک مقاله را تعیین کند میخواهیم زمانی که یک مقاله اضافه کرد اون مقاله
			به صورت دیفالت نویسندش بشه همون نویسنده‌ای که مقاله را نوشه و وضعیت هم draft باشه به صورت خودکار برای این کار از
			از یک میکسین و یک متد \متن‌لاتین{form valid} استفاده میکنیم و مشخص میکنیم اگه کاربر نویسنده عادی بود به صورت خود 
			کار نویسنده را برابر نویسنده مقاله قرار بده و از وضعیت draft استفاده کن بعد از میکسین داخل ویو مربوطه استفاده 
			میکنیم 
			
			برای اینکه بتونیم مقدار فیلد هارو داخل فرم تغییر بدیم از \متن‌لاتین{commit=False} استفاده میکنیم. 
		
		\begin{figure}[h!]
			\includegraphics[width=\linewidth]{mehran-tarif-course-code-pic/FormValidMixinPic.jpg}
		\end{figure}
	
	
	
\end{document}